%!TEX root = ../thesis.tex
%!TEX spellcheck
%\subsubsection{Abstract}
This thesis presents a combined experimental and computational evaluation of the physical-organic properties of butadiyne-linked porphyrin oligomers. The principal result from the thesis is the synthesis and characterisation of the largest aromatic and antiaromatic systems to date, in the form of an oxidised [6]-porphyrin nanoring, with diameter \SI{2.4}{\nano\meter}. This large electronically coherent system provides insight into the connection between aromatic ring currents and persistent currents in metal and semiconductor mesoscopic rings. 

\autoref{ch:intro} briefly reviews the concepts used in the remainder of the thesis, with a particular focus on aromaticity.

In \autoref{ch:dimer}, the barrier to inter-porphyrin torsional rotation in a butadiyne-linked porphyrin dimer is determined computationally and experimentally to be \SI{3}{\kilo\joule\per\mole}. The barrier height is closely related to the resonance delocalisation energy between the porphyrin subunits. 

In \autoref{ch:hexacat} we show that by oxidising a butadiyne-linked [6]-porphyrin nanoring to its 4+ and 6+ oxidation states, the nanoring becomes antiaromatic and aromatic respectively. In contrast, the neutral oxidation state exhibits only local aromaticity for the six porphyrin units. The 12+ cation can also be generated, and exhibits local antiaromaticity for each porphyrin unit. The characterisation of (anti)aromaticity employs NMR and computational techniques. 

In \autoref{ch:radcat}, the properties of cation radicals of linear and cyclic porphyrin oligomers are explored. Cations generated by spectroelectrochemistry are measured by optical spectroscopies, and chemically generated radical monocations are examined by cw/pulsed EPR spectroscopies. EPR and optical spectroscopies agree that the dimer monocation radical is fully delocalised, in Robin-Day Class III, whereas the monocations of longer oligomers are localised over 2--3 porphyrin units (Class II).

In \autoref{ch:excited-arom}, photophysical and computational investigations into excited state aromaticity in porphyrin nanorings are presented. The computational results suggest the presence of aromaticity in the triplet excited states, but experiment fails to convincingly demonstrate the effect. 

Computational results in \autoref{ch:truncnano} show that a butadiyne linked [6]-porphyrin nanoring in which one butadiyne (\ce{C#C-C#C}) is truncated to an alkyne (\ce{C#C}) exhibits a reversal of aromaticity and antiaromaticity in its oxidised states, compared to the all-butadiyne linked nanoring, consistent with H\"uckel’s law.
